\documentclass[11pt]{article}
\usepackage[utf8]{inputenc}
\usepackage{amsfonts}
\usepackage{amsmath}
\title{Mathematics for Computer Science - Problem set 01}
\author{Ngoc Tan Pham - 19520925 }
\date{September 28$^{th}$ 2020}

\begin{document}

\maketitle

\paragraph{Problem 1:}
\label{sec::problem1}We have $\textbf{u}$ and $\textbf{v}$ are $\textbf{orthogonal unit vectors}$. Show that $\textbf{u + v}$ is orthogonal to $\textbf{u - v}$.
\\[0.4cm]
Let $u, v \in \mathbb{R}^n$. Since $u$ and $v$ are unit vector, we assume:
\begin{center}
    ${u}$ = $\begin{bmatrix} 1,&0, &0, &... &0 \end{bmatrix}$\\[0.2cm] 
    ${v}$ = $\begin{bmatrix} 0,&1, &0, &... &0 \end{bmatrix}$ 
\end{center}
As to prove $u$ + $v$ is orthogonal to $u$ - $v$, we have to show that the dot product between $u$ + $v$ and $u$ - $v$ is 0\\[0.2cm]
We have:
\begin{center} 
    $u + v = \begin{bmatrix} 1, &1, &0, &0, &... &0 \end{bmatrix}$ \\[0.2cm]
    $u - v = \begin{bmatrix} 1,&-1, &0, &... &0 \end{bmatrix}$ \\[0.2cm]
\end{center}
And
$$\sum\limits_{i=1}^{n} (u + v)_{i}\cdot(u - v)_{i}$$
$$= (1 \cdot 1) + [1 \cdot (-1)] + (0 \cdot 0) + \cdots + (0 \cdot 0) = 0$$
Since the dot product between $u$ + $v$ and $u$ - $v$ is $0$, then $u$ + $v$ is orthogonal to $u$ - $v$ 




\paragraph{Problem 2:}
\label{sec::problem2} Prove that nullspaces $\textbf{N}$($A^\top A$) = $\textbf{N}(A)$
\\[0.4cm]
Let $A$ denote a $\mathbb{R}^{m\times n}$ matrix
\begin{center}
$A_{mn} = 
    \begin{pmatrix}
    a_{11} & a_{12} & \cdots & a_{1n} \\
    a_{21} & a_{22} & \cdots & a_{2n} \\
    \vdots  & \vdots  & \ddots & \vdots  \\
    a_{m1} & a_{m2} & \cdots & a_{mn} 
    \end{pmatrix}
$
\end{center}
$$\Rightarrow A^\top = 
 \begin{pmatrix}
  a_{11} & a_{21} & \cdots & a_{m1} \\
  a_{12} & a_{22} & \cdots & a_{m2} \\
  \vdots  & \vdots  & \ddots & \vdots  \\
  a_{1n} & a_{2n} & \cdots & a_{mn} 
 \end{pmatrix}
$$
Since $\textbf{N}$($A$) is the null space of $A$. Take $x \in \textbf{N}(A)$, we have 
$$A \times x = 0$$
Multiply two sides of the equation by $A^\top$, we have
$$A^\top \times A \times x  = A^\top \times 0$$
$$\Leftrightarrow A^\top \times A \times x = 0$$
$$\Rightarrow x \in \textbf{N}(A^\top A)$$
Then
\begin{equation}
    \textbf{N}(A) \subseteq \textbf{N}(A^\top A)
\end{equation}
Moreover $\textbf{N}(A^\top A)$ is the null space of $A^\top A$. Take $x \in \textbf{N}(A^\top A) $
$$A^\top A \times x = 0$$
Multiply two sides of the equation by $x^\top$, we have
$$x^\top \times A^\top A \times x = 0 $$
$$\Leftrightarrow (A \times x)^\top \times (A \times x) = 0 $$
$$\|Ax\| = 0 $$
$$\Rightarrow Ax = 0 $$
\begin{equation}
    \Rightarrow A^\top A \subseteq A
\end{equation}
From $(1)$ and $(2)$, we have
$ 
\begin{cases}
    A \subseteq A^\top A\\
    A^\top A\subseteq A
\end{cases}
$\\[0.2cm] 
$\Rightarrow\textbf{N}(A) = \textbf{N}(A^\top A)$\\[0.2cm]
$\rightarrow$ Q.E.D







\section*{Problem 3}
We know for sure that a symmetric matrix must be a square one\\[0.2cm]
Let $A$ denote a $\mathbb{R}^{n\times n}$ matrix
$$ 
 \begin{pmatrix}
  a_{11} & a_{21} & \cdots & a_{n1} \\
  a_{12} & a_{22} & \cdots & a_{n2} \\
  \vdots  & \vdots  & \ddots & \vdots  \\
  a_{1n} & a_{2n} & \cdots & a_{nn} 
 \end{pmatrix}
$$
We have the column space and the row space of $A$: $\textbf{C}(A)$ and $\textbf{C}(A^\top)$, respectively\\[0.2cm]
$\textbf{C}(A)$ is the combination of all the columns in $A$ and $\textbf{C}(A^\top)$ is the combination of all the rows in $A$, also.\\[0.2cm]
Hence
$$
\textbf{C}(A) = \textbf{C}(A^\top)
$$
However, the elements in $A$ can be linearly independent of each other, which leads to the asymmetry of $A$. Or in other words, $A \neq A^\top$\\[0.2cm]
Indeed, take a closer look to the example:\\[0.2cm]
Let $A$ be given a 2$\times$2 matrix
$$ A =
\begin{pmatrix}
1 & 1 \\ 0 & 1
\end{pmatrix}
$$
The column space of $A$ is
$$
\textbf{C}(A) = \begin{bmatrix} 1 & 1 \\ 0 & 1 \end{bmatrix} 
$$
and the row space of $A$ is
$$
\textbf{C}(A^\top) = \begin{bmatrix} 1 & 0 \\ 1 & 1 \end{bmatrix}
$$
Thus
$$
\textbf{C}(A) = \textbf{C}(A^\top)
$$
However, since $A \neq A^\top$, then $A$ does not have to be symmetric if $\textbf{C}(A) = \textbf{C}(A^\top)$

\section*{Problem 4}
Let A denote a $2\times$2 matrix
$$ A =  
\begin{pmatrix}
2 & 4 \\ -1 & -2
\end{pmatrix}
$$
Since
$$A \times \textbf{N}(A) = \begin{pmatrix} 2 & 4 \\ -1 & -2 \end{pmatrix} \times \begin{bmatrix} -2 \\ 1 \end{bmatrix} = 0$$
Then the null space of $A$ is $$\textbf{N}(A) = \begin{bmatrix} -2 \\ 1 \end{bmatrix}$$
However
$$A^2 = \begin{pmatrix} 2 & 4 \\ -1 & -2 \end{pmatrix} = \begin{pmatrix} 2 & 4 \\ -1 & -2\end{pmatrix} \times \begin{pmatrix}2 & 4 \\ -1 & -2 \end{pmatrix} = \begin{pmatrix} 0 & 0 \\ 0 & 0\end{pmatrix}$$
which leads to the null space of $A^2$ is
\begin{center}
    $\textbf{N}(A^2) =  \begin{bmatrix} x_1 \\ x_2 \end{bmatrix}$ (in which $x_1, x_2 \in \mathbb{R}$)
\end{center}
Hence $$\textbf{N}(A) \subseteq \textbf{N}(A^2)$$which means it is not $\emph{always}$ true that $\textbf{N}(A) = \textbf{N}(A^2)$ 
\end{document}

